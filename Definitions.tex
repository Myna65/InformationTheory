\documentclass[a4paper,12pt]{article}
\usepackage[utf8]{inputenc}
\usepackage[french]{babel}
\usepackage[T1]{fontenc}
\begin{document}
\section{Définitions}

\paragraph{Channel p.77}
	We shall consider a channel to be specified if we 
	know the following elements
	\begin{enumerate}
		\item the input alphabet $A$
		\item the output alphabet $B$
		\item the probability $\nu_x(S)$ that $y$ received 
		      when a given $x$ is transmitted belongs to 
		      the set $S \in F_B$.
	\end{enumerate}

\paragraph{Stationary p. 77}
	We shall call the channel $[A,\nu_x,B]$ stationary if, 
	for all $x\in A^I$ and $\in F_B$
	\[\nu_{Tx}(TS)=\nu_x(S)\]

\paragraph{Channel without anticipation p. 77}
	We shall call a channel, channel without anticipation 
	if the distribution of $y_n$ is indepedent of the transmitted 
	signal that are transmitted  after $x_n$

\paragraph{Memory pp. 77-78}
	We shall call the memory of a channel the smallest number
	$m$ that all $y_n$ depends only on $(x_n, ..., x_{n-m})$.
	
\paragraph{Ergodic capacity p.91}
	We shall call the capacity of a channel the least upper bound
	of $R(X,Y) = H(X)-H_Y(X)$.
	
\paragraph{Source p. 45}
	We shall consider a source to be specified if we 
	know the following elements
	\begin{enumerate}
		\item an alphabet $A$.
		\item a probability $\mu(S)$ defined for all $S\in F_A$.
	\end{enumerate}
	We denotate this by $[A,\mu]$
	
	

\end{document}
